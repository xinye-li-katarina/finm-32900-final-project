\section*{Table 3: Pairwise Correlations}

Table 3 shows the pairwise correlations between various capital ratios and macroeconomic indicators over the sample period from Q1 1970 to Q4 2012, using the data from CRSP-Compustat, Datastream, and several macroeconomic databases.

The process of constructing Table 3 begins with the data from Table 2, `prim\_deal\_merge\_manual\_data\_w\_linktable,' which assembles the comprehensive dataset for the primary dealers. This involves gathering the financial metrics—total assets, book equity, and market equity — to calculate the market and book capital ratios.  The dataset starts a year before 1970 to set the stage for the subsequent factor calculation and growth rate analyses.

A key aspect of this paper is its engagement with the findings from Adrian-Etula-Muir (AEM), a paper highlighting the predictive power of broker-dealer leverage on stock market returns. Table 3 illustrates this engagement by including the AEM leverage factor as one of the major correlation variables. However, when we pulled the current FRED data, the numbers did not match those from the AEM paper. For instance, the total financial assets for the end of 2010.4Q, the AEM paper shows 2,075.1 billion dollars, whereas the current fred data shows 4,312.8 billion dollars. To account for the change in the metrics, we downloaded and used the dataset previously released for 1970-2012, which aligns with the AEM paper numbers.

After the data preparation, we calculated the capital ratios and transformed these ratios into analytical factors, employing AR(1) innovations. Market and book capital factors were defined as AR(1) innovations to the capital ratio, scaled by the lagged capital ratio. The AEM metrics were defined separately; the AEM implied capital as the inverse of broker–dealer book leverage from Flow of Funds used in AEM, and the AEM leverage factor (LevFac) as the seasonally adjusted growth rate in broker–dealer book leverage.

Next, we processed macroeconomic indicators such as earnings-to-price ratio (E/P), unemployment rate, financial conditions index, Real GDP and GDP growth, market excess returns, and market volatility. The E/P ratio is downloaded from a Shiller dataset, with the URL containing the most recently updated data. The market excess return is fetched from Fama-French research datasets, calculated as SPY returns minus the risk-free rate. Like the ratio and factor datasets, the macro dataset begins from a year before 1970 for the subsequent factor and growth rate computations.

Finally, we calibrated Panel A and B correlations to examine the relationships between the intermediary capital ratios and macroeconomic variables. Panel A focuses on the levels of financial ratios and macroeconomic variables, and Panel B focuses on the factors derived from the financial ratios and their growth rates. We summarize our findings into tables for Panel A and B, alongside a figure illustrating how the capital ratios have shifted over time. All time series are standardized to zero mean and unit variance for illustration. We get our final table, which is converted to LaTeX and outputted to a .tex file.