\section*{Table 3: Pairwise Correlations}

Table 3 was compiled by Hang Yu, which shows the pairwise correlations between various capital ratios and macroeconomic indicators over the sample period from Q1 1970 to Q4 2012, using data from NYFED, NIC, CRSP-Compustat, Datastream, and several macroeconomic databases.

\par
The process of constructing Table 3 begins with the data from Table~2. Pulled the list of primary dealers from 1970--current from NYFED. Located the companies' RSSDIDs by looking up their names in the NIC (National Information Center). Then obtained the holding company's RSSDID from the Compustat database, and for any RSSDIDs couldn't find there, conducted manual Google searches, referencing The New York Times to confirm dates of ownership transfer. Finally, matched the appropriate \texttt{gvkey} to grant access to financial information.

\par
\textbf{Metrics to be replicated:}

\textit{Market Capital Ratio} is a firm's market equity divided by the sum of market equity and book debt, indicating how much of total capital is represented by market value.

\textit{Book Capital Ratio} is book equity over (book debt plus book equity), capturing the proportion of book equity in total capital.

\textit{AEM Leverage Ratio} is defined as financial assets divided by (financial assets minus financial liabilities), reflecting leverage tied to accounting-based earnings management.

\textit{Market Capital Factor} originates from running an \( AR(1) \) regression on the \( market\_cap\_ratio \). The resulting residual is then scaled by the ratio's lagged value, highlighting unexpected variations in market capitalization.

\textit{Book Capital Factor} follows the same procedure for the \( book\_cap\_ratio \), measuring shocks specific to book-based capital structure.

\textit{AEM Leverage Factor} is constructed by taking the log of the AEM leverage ratio, differencing it, and removing quarterly fixed effects through a regression. The residual signifies leverage innovations beyond seasonal patterns.

\par
\textbf{Macroeconomic Indicators:}

\textit{Unemployment Rate} reflects the percentage of the labor force currently without work, offering insight into overall economic health and labor market slack.

\textit{Financial Conditions Index (NFCI)} captures the ease or tightness of financial markets, encompassing credit spreads, leverage, and funding costs to gauge the financial environment.

\textit{Real GDP} measures the inflation-adjusted output of goods and services within an economy, serving as a comprehensive indicator of economic activity and growth.

\textit{Earnings/Price (E/P) ratio} is derived from the inverse of the Shiller CAPE (Cyclically Adjusted Price-to-Earnings) index, capturing long-term valuation trends in equity markets; higher values may suggest potentially higher future returns.

\textit{Market Excess Return} uses the Fama--French market factor, often representing the stock market's return above a risk-free benchmark, signaling market risk appetite or sentiment.

\textit{Market Volatility} is measured by the standard deviation of returns on a value-weighted market index over a given period, gauging fluctuations in stock prices and investor uncertainty.

\par
Finally, compute the correlations between the macroeconomic data and three ratios and factors to replicate \textbf{Panel~A} and \textbf{Panel~B} respectively.
